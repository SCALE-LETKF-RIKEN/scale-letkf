\section{Observation departure file format}
\label{sec:obsdep-file-format}
Observation departure statistics output is enabled when the namelist parameter \verb|OBSDEP_OUT| of \verb|PARAM_LETKF_MONITOR| is set.   

The common obsdep file format used in SCALE-LETKF consists of 12-record elements of 4-byte floating-point value for each observation. Note that the access type is \textit{sequential} and not \textit{direct}. The following is a part of the subroutine \verb|write_obs_dep| in \verb|scale/common/common_obs_scale.f90| (simplified for brevity). 

\begin{Verbatim}[frame=lines, framesep=3mm, label=scale/common/common\_obs\_scale.f90]
  OPEN(iunit,FILE=cfile,FORM='unformatted',ACCESS='sequential')
  DO n=1,obs%nobs
    wk(1) = real(obs(set(n))%elm(idx(n)), r_sngl)
    wk(2) = real(obs(set(n))%lon(idx(n)), r_sngl)
    wk(3) = real(obs(set(n))%lat(idx(n)), r_sngl)
    wk(4) = real(obs(set(n))%lev(idx(n)), r_sngl)
    wk(5) = real(obs(set(n))%dat(idx(n)), r_sngl)
    wk(6) = real(obs(set(n))%err(idx(n)), r_sngl)
    wk(7) = real(obs(set(n))%typ(idx(n)), r_sngl)
    wk(8) = real(obs(set(n))%dif(idx(n)), r_sngl)
    wk(9) = real(qc(n), r_sngl)
    wk(10) = real(omb(n), r_sngl)
    wk(11) = real(oma(n), r_sngl)
    wk(12) = real(spr(n), r_sngl)
    WRITE(iunit) wk
  END DO
\end{Verbatim}

The 12 elements of the observation departure data array are as follows. 

\begin{table}[h]
    \centering
    \begin{tabular}{|l|l|}
        \hline
        Record \# & Description \\ \hline\hline
        1 & Observation variable (integer code defined in \verb|common_obs_scale.f90|)\\ \hline
        2 & Longitude (degree) \\ \hline
        3 & Latitude (degree)  \\ \hline
        4 & Vertical level (hPa or meter, see \hyperref[sec:vertical-coordinate]{Notes on vertical coordinate}) \\ \hline
        5 & Observed value \\ \hline
        6 & Observation error standard deviation \\ \hline
        7 & Observation type (integer code defined in \verb|common_obs_scale.f90|) \\ \hline
        8 & Difference of observation time from the target time of data assimilation (second) \\ \hline
        9 & QC flag (integer code defined in \verb|common_obs_scale.f90|) \\ \hline
        10 & O-B \\ \hline
        11 & O-A \\ \hline
        12 & First guess ensemble spread in the observation space \\ \hline
    \end{tabular}
    \caption{Observation departure data elements}
    \label{tab:obsdep-data-elements}
\end{table}
